\documentclass[./main.tex]{subfiles}
\include{mathnotes}
\include{cat_preamble}
\include{url}

\begin{document}

\section{Condensed sets}

\begin{remark}
This section roughly corresponds to \cite[Lecture 1]{scholze1} as well as \cite[Lectures 1-3]{masterclass}. 
\end{remark}

Condensed sets are an alternate formulation of ``geometric objects'' that are better behaved than topological spaces in some sense. In particular, condensed sets provide a suitable setting for the homological algebra of topological rings and modules. This is desirable because abelian group objects in $\Top$ do not form an abelian category. Consider the following morphism in the category of topological abelian groups
$$(\RR, \text{ discrete topology}) \rightarrow (\RR, \text{ natural topology})$$
whose underlying set map is the identity on $\RR$. Its kernel and cokernel are both trivial, yet this is not an isomorphism. In the category of condensed sets, the above morphism (suitable interpreted) has a nontrivial cokernel measuring the difference in topology.

There have already been some attempts to define cohomology of topological algebraic objects. For example, \textit{continuous group cohomology} considers a topological $G$ acting continuously on a topological abelian group, but there was no site-theoretic justification for its definition. 

Roughly, the category of condensed sets is the ($\Set$-valued) sheaf category on a site of profinite sets. This category has two obvious advantanges over $\Top$. First, recall that $\Top$ is not a topos while a sheaf category is (by definition) a Grothendieck topos so intuitionistic logic can be interpreted in it. Second, abelian group objects in a sheaf topos form an abelian category hence we can perform homological algebra. On the other hand, this definition ignores set-theoretic issues and offers no motivation for considering profinite sets. Thus, the goal of this section is to explain these questions in detail. 

Condensed sets will be sheaves on some Grothendieck site whose objects are special topological spaces called \textit{profinite sets}. In order for a sheaf category to be a topos, the site must be essentially small. Thus it is reasonable to put a bound on the cardinality of topological spaces considered. 

\begin{definition}
A cardinal $\kappa$ is said to be a \textbf{strong limit cardinal} if for all $\lambda < \kappa$, it holds that $2^\lambda < \kappa$. Such cardinals can easily be constructed and in most cases it suffices to consider $\kappa = \beth_\omega$ in the notation of \cite[Remark 1.3]{scholze1}.
\end{definition}

Most topological spaces in practice are compactly generated, which essentially means that properties about them can be checked on compact Hausdorff subspaces. Hence we begin by considering a site of compact Hausdorff spaces. 

\begin{definition}
For a cardinal $\kappa$, let $\CHaus_\kappa$ be the full subcategory of $\Top$ spanned by compact Hausdorff spaces $X$ with $|X| < \kappa$. 
\end{definition}

\begin{definition}\label{covering}
For $Y \in \CHaus_\kappa$, a set of morphisms $\{f_i : X_i \rar{} Y\}$ in $\CHaus_\kappa$ is a \textbf{covering} of $Y$ if it is a finite set of jointly surjective maps. 
\end{definition}

It is easy to check that this defines a coverage which generates a Grothendieck topology on $\CHaus_\kappa$. The following shows that this topology is not too strong.

\begin{proposition}[\cite[Proposition 1.7]{scholze1}]
There is a faithful functor
$$\Top \rightarrow \Sh(\CHaus_\kappa)$$
defined by
$$X \mapsto \Hom_{\Top}(\_,X)$$
This is fully faithful on the subcategory of $\kappa$-compactly generated spaces. 
\end{proposition}
\begin{proof}
It is obvious that there is such a faithful functor landing in $\Psh(\CHaus_\kappa)$. It remains to argue for each topological space $X$ that $\Hom_{\Top}(\_,X)$ is a sheaf on $\CHaus_\kappa$. Let $\{f_i : Y_i \rar{} Z\}$ be a covering of $Z$ in $\CHaus_\kappa$, and we must check that
\begin{center}
\begin{tikzcd}
\Hom(Z,X) \ar[r] & \prod_i \Hom(Y_i, X) \ar[r, yshift = 0.7ex] \ar[r, yshift = -0.7ex] & \prod_{i,j} \Hom(Y_i \times_Z Y_j, X)
\end{tikzcd}
\end{center}
is an equalizer diagram. Consider an element in the equalizer represented by a collection of maps $g_i : Y_i \rar{} X$. Define a set map $g : Z \rar{} X$ by $g(z) = g_i(y_i)$ for any preimage $y_i \in Y_i$ of $z$ along any $f_i$. This is well-defined by the equalizer condition. To see that $g$ is closed, note for any $C \subseteq X$ that $g^{-1}(C) = \cup_i f_i(g_i^{-1}(C))$. Since the covering family is finite, and a continuous map of compact Hausdorff spaces preserve closed sets, it follows that $g^{-1}(C)$ is closed whenever $C$ is closed. Thus $g$ is continuous. 
\end{proof}

It turns out that $\CHaus_\kappa$ has ``enough projective objects'' in the following sense. First, a topological space $T$ is said to be \textit{projective} if it satisfies the lifting property
\begin{center}
\begin{tikzcd}
 & X \ar[d, twoheadrightarrow]
 T \ar[r] \ar[ru, dashed] & Y
\end{tikzcd}
\end{center}
for any surjection $X \twoheadrightarrow Y$. Having \textit{enough} projective object means that for any $X \in \CHaus_\kappa$, there is a projective object $T$ in $\CHaus_\kappa$ and a surjection
$$T \twoheadrightarrow X$$
In fact, we can take $T = \beta X$ to be the Stone-\v{C}ech compactification of the set underlying $X$ (this is a slight abuse of notation as $\beta X$ usually refers to the Stone-\v{C}ech compactification of $X$ itself). Formally, the Stone-\v{C}ech compactification is the left adjoint to the forgetful functor $\Top \rightarrow \CHaus$. More explicitly, for a discrete space $X$ its Stone-\v{C}ech compactification can be constructed by putting a topology on the set of ultrafilters on $X$. 

\begin{definition}
A \textbf{filter} $\mathcal{F}$ on a set $X$ is a family of subsets of $X$ such that
\begin{itemize}
\item (Upward closed). If $A \in \mathcal{F}$ and $A \subseteq B$, then $B \in \mathcal{F}$.
\item ($\cap$-closure). If $A, B \in \mathcal{F}$, then $A \cap B \in \mathcal{F}$. 
\end{itemize}
A filter $\mathcal{F}$ is \textbf{proper} if $\mathcal{F} \neq 2^X$. An \textbf{ultrafilter} is a nonempty, proper filter $\mathcal{F}$ such that for all $A \subseteq X$, either $A \in \mathcal{F}$ or $X \setminus A \in \mathcal{F}$. 
\end{definition}

The topology on the set of all ultrafilters on $X$ is generated by $\{\mathcal{F} \mid U \in \mathcal{F}\}$ for $U \subseteeq X$. Moreover, the set of all ultrafilters has cardinality at most $2^{2^|X|} < \kappa$ assuming that $|X| < \kappa$, so $\beta X$ is an object of $\CHaus_\kappa$. The spaces $\beta X$ admit the following characterization. 

\begin{definition}
A topological space $Y$ is \textbf{extremally disconnected} if the closure of an open set is also open. Equivalently, $Y$ is extremally disconnected if any surjection $Y' \rar{} Y$ from a compact Hausdorff $Y'$ splits. 
\end{definition}

Every $\beta X$ is extremally disconnected (and compact Hausdorff), and conversely every extremally disconnected compact Hausdorff space is of the form $\beta X$ for some discrete space $X$. Thus, by definition of the topology on $\CHaus_\kappa$, any sheaf is uniquely determined by its values on extremally disconnected compact Hausdorff spaces. When restricting to such objects, the sheaf condition becomes a lot simpler : 

\begin{lemma}
$F \in \Psh(\CHaus_\kappa)$ is a sheaf if and only if
\begin{itemize}
\item $F(\emptyset) = *$
\item $F(Y_1 \amalg Y_2) \iso F(Y_1) \times F(Y_2))$ for every extremally disconnected compact Hausdorff spaces $Y_1$ and $Y_2$.
\end{itemize}
\end{lemma}

Here are some remarks on extremally disconnected compact Hausdorff spaces. 

\begin{rem}\hfill
\begin{itemize}
\item Extremally disconnected compact Hausdorff spaces are not closed under fiber products; they are not closed under products and taking closed subsets. 
\item Every projective compact Hausdorff space is a retract of some extremally disconnected compact Hausdorff space.
\item Axiom of Choice may be needed to produce a non-principal ultrafilter, that is, a point of $\beta X$ not contained in $X$. 
\end{itemize}
\end{rem}

As we can see, extremally disconnected compact Hausdorff spaces are a bit unwieldy to handle as a site. Examples are hard to explicitly write down, and they are not closed under finite limits. Thus it makes sense to consider objects of $\CHaus_\kappa$ more general than extremally disconnected ones. 

\begin{definition}
A topological space is \textbf{totally disconnected} if the only connected subsets are the singletons. Such a space is also called a \textbf{profinite set}.
\end{definition}

As an example, $\beta X$ is a profinite set that can be interpreteed as a profinite completion of $X$. Let $J$ be the diagram category whose objects are surjections $X \rar{} S$ to a finite set $S$, and a morphism between objects is given by a commuting triangle. Then there is a homeomorphism
$$\beta X \iso \underset{S \in J}{\lim} S$$
where the limit on the RHS is taken in $\Top$. Note that $J$ is a filtered category, so $\beta X$ is really a ``pro-object'' built out of finite sets. More generally, every extremally disconnected space is a profinite set. Profinite sets are also closed under all limits, in particular fiber products. This allows us to view the subcategory $*_{\kappa-\proet}$ of $\CHaus_\kappa$ spanned by profinite sets, as a site with the topology induced from $\CHaus_\kappa$. 

We now arrive at the main definition. 

\begin{definition}
The category of $\kappa$-condensed sets is
$$\Cond = \Sh(*_{\kappa-\proet})$$
(we suppress denoting the cardinality $\kappa$). 
\end{definition}

\section{Condensed abelian groups}

\begin{remark}
This section roughly corresponds to \cite[Leture 2]{scholze1} and \cite[Lectures 4-]{masterclass}
\end{remark}

Since $\Cond$ is a topos, the category $\Ab(\Cond)$ of abelian group objects in $\Cond$ is an abelian category. Moreover, it satisfies additional axioms that make it further resemble the category $\Ab$ of (classical) abelian groups. Henceforth we call objects of $\Ab(\Cond)$ condensed abelian groups.

\begin{theorem}[\cite[Theorem 1.10]{scholze1}]
$\Ab(\Cond)$ is an abelian category that satisfy in addition Grothendieck's axioms AB3, AB4, AB5, AB6, AB3$^*$ and AB$4^*$. 
\end{theorem}

Most abelian categories fail to satisfy AB$4^*$ and AB6. The reason why $\Ab(\Cond)$ satisfy these axioms is because $\Ab(\Cond)$ admits compact projective generators. 

\begin{definition}
An object $T$ of a category $\C$ is \textbf{$\kappa$-compact} if $\Hom_\C(T,\_)$ commutes with $\kappa$-filtered colimits. An object is compact if it is $\omega$-compact.
\end{definition}

\begin{lemma}
If an object $T$ of an abelian category $\mathcal{A}$ is compact projective, then $\Hom_\mathcal{A}(T,\_)$ preserves all limits and colimits. 
\end{lemma}

\begin{definition}
An object $T$ of a category $\C$ is a \textbf{generator} if $\Hom_\C(T,\_) : \C \rightarrow \Set$ is faithful. 
\end{definition}

\begin{example}
The abelian category $\Mod_R$ of left modules over an associative ring $R$ admits a compact projective generator $R$. 
\end{example}

\begin{proposition}
$\Ab(\Cond)$ has enough compact projectives, meaning that every object $M$ admits an epimorphism 
$$\oplus_i P_i \rar{} M$$
from some set of compact projectives $\{P_i\}_i$. In particular, the compact projectives can be taken to be $\Z[T]$, where $T$ is a representable corresponding to an extremally disconnected compact Hausdorff space, and $\Z[\_]$ denotes the left adjoint to the forgetful functor
$$\Cond \rightarrow \Ab(\Cond)$$
\end{proposition}

\bibliography{/bibs/condensed}

\end{document}